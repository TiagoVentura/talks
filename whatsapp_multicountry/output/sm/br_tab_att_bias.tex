\begin{table}[H]

\caption{\label{tab:tab:br_ghanem}Brazil Sample Internal Validity Test for Primary Outcomes and Baseline Variables}
\centering
\resizebox{\linewidth}{!}{
\begin{threeparttable}
\begin{tabular}[t]{>{\raggedright\arraybackslash}p{7cm}rlrrrrr}
\toprule
\multicolumn{1}{c}{ } & \multicolumn{2}{c}{Attrition Bias} & \multicolumn{4}{c}{Mean baseline outcome by group} & \multicolumn{1}{c}{Test of internal validity} \\
\cmidrule(l{3pt}r{3pt}){2-3} \cmidrule(l{3pt}r{3pt}){4-7} \cmidrule(l{3pt}r{3pt}){8-8}
Outcome & C & Differential & TR & CR & TA & CA & pvalue\\
\midrule
Age & 0.87 & 0.0317 (p-value = 0.1236) & 2.50 & 2.51 & 2.63 & 2.56 & 0.95\\
Gender & 0.87 & 0.0317 (p-value = 0.1236) & 0.35 & 0.35 & 0.44 & 0.41 & 0.94\\
Education & 0.87 & 0.0317 (p-value = 0.1236) & 4.49 & 4.52 & 4.30 & 4.46 & 0.40\\
Income & 0.87 & 0.0317 (p-value = 0.1236) & 5.00 & 5.13 & 4.30 & 3.98 & 0.38\\
WP:Daily time & 0.87 & 0.0317 (p-value = 0.1236) & 3.79 & 3.82 & 3.51 & 3.41 & 0.89\\
News Consumption: General & 0.87 & 0.0317 (p-value = 0.1236) & 3.11 & 3.12 & 2.51 & 2.69 & 0.78\\
News Consumption: Social Media Apps & 0.87 & 0.0317 (p-value = 0.1236) & 0.91 & 0.87 & 0.88 & 0.81 & 0.11\\
False News Exposure & 0.87 & 0.0317 (p-value = 0.1236) & 2.42 & 2.36 & 2.21 & 2.39 & 0.60\\
Affective Polarization & 0.87 & 0.0317 (p-value = 0.1236) & 0.19 & 0.22 & 0.71 & 0.14 & 0.03\\
\bottomrule
\end{tabular}
\begin{tablenotes}
\item \textit{Note: } 
\item Column 1 reports the attrition rate for control, and Column 2 reports the differential attrition rate between treatment and control, with the corresponding p-value testing for difference in attrition between the groups (	extit{differential attrition}). Columns 3-6 present the mean baseline outcome for treatment respondents (TR), control respondents (CR), treatment attritters (TA), and control attritters (CA), respectively. Column 7 reports the p-value of the hypothesis test with two equality restrictions (	extit{selective attrition}
\end{tablenotes}
\end{threeparttable}}
\end{table}
