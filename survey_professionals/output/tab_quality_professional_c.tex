\begin{table}[H]

\caption{\label{tab:tab:qualityc}Response quality of survey professionals vs. non-professionals (professionals = more than 50 percent visits to survey sites)}
\centering
\resizebox{\linewidth}{!}{
\begin{threeparttable}
\begin{tabular}[t]{>{\raggedright\arraybackslash}p{7cm}lllllllll}
\toprule
\multicolumn{1}{c}{ } & \multicolumn{3}{c}{Facebook} & \multicolumn{3}{c}{Lucid} & \multicolumn{3}{c}{Yougov} \\
\cmidrule(l{3pt}r{3pt}){2-4} \cmidrule(l{3pt}r{3pt}){5-7} \cmidrule(l{3pt}r{3pt}){8-10}
 & Professionals &  & Non-professionals & Professionals &  & Non-professionals & Professionals &  & Non-professionals\\
\midrule
Straightliner (\%) & 4.8 (4.8) &  & 1.1 (0.4) & 2.4 (0.5) & \circ & 1.1 (0.3) & 3.9 (2.2) &  & 4.4 (0.7)\\
Survey duration (median seconds) & 833 (143.578) &  & 823 (1084.962) & 1116 (295.018) &  & 1140 (387.864) & 1773 (11248.516) &  & 1741 (3250.425)\\
\bottomrule
\end{tabular}
\begin{tablenotes}
\item \textit{Note: } 
\item Standard errors in parentheses. Significance of differences between professionals and non-professionals were tested with a Kolgomorov-Smirnoff test for age, chi-squared tests for gender, education and race, and t-tests for all other variables ($\circ$ p < 0.1; $\star$ p < 0.05; $\star\star$ p < 0.01; $\star\star\star$ p < 0.001). Sociodemographic population data from the US Census; personality data from ANES 2016; political variables from ANES 2020. Variables trust, political interest, knowledge and partisanship were recoded to a scale from 0 to 1 to ensure comparability.
\end{tablenotes}
\end{threeparttable}}
\end{table}
