\begin{table}[H]

\caption{\label{tab:tab:qualityall}Response quality of survey professionals vs. non-professionals (professionals = all categories)}
\centering
\resizebox{\linewidth}{!}{
\begin{threeparttable}
\begin{tabular}[t]{>{\raggedright\arraybackslash}p{7cm}lllllllll}
\toprule
\multicolumn{1}{c}{ } & \multicolumn{3}{c}{Facebook} & \multicolumn{3}{c}{Lucid} & \multicolumn{3}{c}{Yougov} \\
\cmidrule(l{3pt}r{3pt}){2-4} \cmidrule(l{3pt}r{3pt}){5-7} \cmidrule(l{3pt}r{3pt}){8-10}
 & Professionals &  & Non-professionals & Professionals &  & Non-professionals & Professionals &  & Non-professionals\\
\midrule
Straightliner (\%) & 4.1 (2.3) & \circ & 0.9 (0.4) & 2.1 (0.4) &  & 1 (0.4) & 6.4 (2.1) &  & 3.9 (0.7)\\
Survey duration (median seconds) & 678.5 (95.844) & \star & 829.5 (1202.851) & 1106 (284.218) & \star\star & 1217 (472.576) & 1750 (8859.647) &  & 1744 (3306.64)\\
\bottomrule
\end{tabular}
\begin{tablenotes}
\item \textit{Note: } 
\item Standard errors in parentheses. Significance of differences between professionals and non-professionals were tested with a Kolgomorov-Smirnoff test for age, chi-squared tests for gender, education and race, and t-tests for all other variables ($\circ$ p < 0.1; $\star$ p < 0.05; $\star\star$ p < 0.01; $\star\star\star$ p < 0.001). Sociodemographic population data from the US Census; personality data from ANES 2016; political variables from ANES 2020. Variables trust, political interest, knowledge and partisanship were recoded to a scale from 0 to 1 to ensure comparability.
\end{tablenotes}
\end{threeparttable}}
\end{table}
